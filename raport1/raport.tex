\documentclass[12pt, a4paper, oneside]{mwart} % Z automatu 10pt w przypisach
\usepackage[utf8]{inputenc} % Znaki diakrytyczne z klawiatury
\usepackage[OT4]{fontenc} % OT4 ponizej nie dzialalo
\usepackage[plmath,MeX]{polski} % Ponoc lepsza polonizacja LaTeXa
%\usepackage[dvips]{graphicx}
\usepackage[pdftex]{color,graphicx} % Grafika w PDFowej formie
%\usepackage{dcolumn} % Wyrownywanie przecinka w tabelach
%\newcolumntype{d}[1]{D{.}{,}{#1}} % Typ kolumny do wyrownywania
%\usepackage{threeparttable} % Coby ladnie podpisac tabelki
%\usepackage{rotating} % for sidewaystable
%\usepackage{subfig}

\usepackage[pdftex]{hyperref} % Zarzadza hiperlaczami w dokumencie, ostatni w preambule, dvips/pdftex zaleznie od wyjscia

\begin{document}
\title{Zaawansowane Modelowanie Symulacyjne [234060-0723]\\ 
\bigskip
PiTU~S.A.\\
Symulacja portfela ubezpieczeń komunikacyjnych}
\author{Anna Chojnacka, 65514 (budowa modelu) \and
Michał Puchalski, 67827 (analiza wrażliwości) \and
Paweł Sadłowski, 68404 (edycja raportu)}
\maketitle

\begin{abstract}
Tutaj streszczenie
\end{abstract}

\section{Opis~organizacji}
PiTU S.A. to~zakład ubezpieczeń specjalizujący~się w~ubezpieczeniach komunikacyjnych. Jego~aktywność jest skoncentrowana w~nieznanym z~nazwy kraju. Firma należy do~liderów silnie skoncentrowanego rynku, jej ubiegłoroczne przychody wyniosły niespełna 843~tys.~PLN. Najsilniejszą presję konkurencyjną wywiera lokalny zakład ubezpieczeń z~dużym udziałem własnościowym skarbu państwa oraz trzy mniejsze filie korporacji zagranicznych.

\section{Opis problemu}
Kraj, w~którym PiTU~S.A. prowadzi swoją działalność notuje rosnącą imigrację z~sąsiedniego Dżydżykistanu. Z~uwagi na~naturalnie wyższą skłonność do~zachowań ryzykownych u~przedstawicieli tej~narodowości firma spodziewa~się wzrostu szkodowości, w~związku z~czym przewiduje konieczność podniesienia składki. Celem przeprowadzonej symulacji jest wyznaczenie odpowiedniej wysokości składki zabezpieczającej towarzystwo ubezpieczeniowe przed niewypłacalnością. Zakres analizy obejmuje także ocenę wrażliwości nadwyżki końcowej oraz prawdopodobieństwa niewypłacalności na~wysokość nadwyżki początkowej oraz pobieranej składki.
\subsection{Szczegółowy scenariusz symulacji}
%\begin{table}
%\caption{Rozkład szkód zgłaszanych przez Dżydżyków}
%\label{l_szkod}
%\begin{tabular}{r|l}
%0&3 437\\
%1&522\\
%2&40\\
%3&2\\
%4&0\\
%5&0\\
%\end{tabular}
%\end{table}
%\begin{table}
%\caption{Wysokość szkód zgłaszanych przez Dżydżyków}
%\label{wys_szkod}
%\begin{tabular}{r|l}
%100&0\\
%200&2\\
%500&27\\
%1 000&52\\
%2 000&115\\
%5 000&203\\
%10 000&106\\
%20 000&42\\
%40 000&14\\
%50 000&0\\
%55 000&0\\
%60 000&1\\
%\end{tabular}
%\end{table}

\begin{table}
\centering
\caption{Rozkład szkód zgłaszanych przez Dżydżyków}
\label{r_szkod}
\begin{tabular}{p{1cm}|l||c||r|l}
\multicolumn{2}{c||}{Liczba~szkód}&\hspace{1cm}&\multicolumn{2}{c}{Wysokość szkody}\\ \hline
0&3 437& \hspace{1cm} & 100&0\\
1&522& \hspace{1cm} & 200&2\\
2&40& \hspace{1cm} & 500&27\\
3&2& \hspace{1cm} & 1 000&52\\
4&0& \hspace{1cm} & 2 000&115\\
5&0& \hspace{1cm} & 5 000&203\\
&&&10 000&106\\
&&&20 000&42\\
&&&40 000&14\\
&&&50 000&0\\
&&&55 000&0\\
&&&60 000&1\\
\end{tabular}
\end{table}

Firma PiTU~S.A. dysponuje historycznymi danymi dotyczącymi liczby oraz wielkości szkód zgłaszanych przez klientów dżydżyckiej narodowości. Przekazane informacje zostały ujęte w tabeli~\ref{r_szkod}. Prezes zarządu spodziewa~się portfela liczącego ok.~100 polis, a~bieżąca nadwyżka wynosi 10~000~PLN. Dział~Aktuarialny PiTU~S.A. rekomenduje przybliżanie liczby szkód przez rozkład Poissona, a~ich wysokości przez rozkład log-normalny. Przedstawione dane nie~dają podstaw do~odrzucenia tych założeń --- do~ich weryfikacji zastosowano odpowiednio: test $\chi^2$ oraz test Kołmogorowa-Smirnowa. W~obu przypadkach p-value znacznie przekraczało 0.9, więc wykorzystanie w~procesie modelowania wymienionych rozkładów jest uzasadnione. Horyzont czasowy analizy został ustalony na~dwa lata.

\subsection{Struktura modelu}
Modelowanym zjawiskiem jest wielkość nadwyżki w~dyspozycji zakładu ubezpieczeniowego w~poszczególnych dniach. W~każdej iteracji symulacji ustalona liczba klientów (100) wykupuje polisę ubezpieczeniową w~trakcie pierwszego roku. Dla każdego z~klientów data zawarcia umowy jest losowana z~rozkładu o~jednakowych prawdopodobieństwach. Następnie dla każdego klienta losowane są: liczba szkód (z~rozkładu Poissona), a~także data wystąpienia (ponownie z~rozkładu o~jednakowych prawdopodobieństwach) oraz wysokość każdej z~nich (rozkład log-normalny) --- o~ile wystąpią. Ostatecznie stan nadwyżki jest obliczany na~każdy kolejny dzień w~horyzoncie czasowym symulacji, z~uwzględnieniem wpływów ze~składek oraz wydatków na~pokrycie szkód. W~przypadku gdy dowolnego dnia suma szkód przekracza wartość dostępnej nadwyżki, dochodzi do~niewypłacalności. W~przeciwnym razie model zwraca stan nadwyżki na~koniec okresu symulacji.

\section{Wyniki analizy}

\subsection{Optymalna wysokość składki}
Krzywa prawdopodobieństwa niewypłacalności przedstawiona na~wykresie ma~kształt hiperboli. Wybór~optymalnej wysokości składki powinien zatem równoważyć niskie prawdopodobieństwo bankructwa oraz niski poziom składki, tak~aby zniechęceni klienci nie~zdecydowali~się na~konkurencyjną ofertę. Punkt załamania krzywej znajduje~się w~okolicach składki wynoszącej 1000~PLN. Prawdopodobieństwo niewypłacalności wynosi wówczas niespełna 10\%, a~co za~tym idzie --- wciąż jest duże. Najmniejszą wysokością składki, dla~której nie~przekracza ono poziomu 1\% jest 1500~PLN.

\subsection{Rentowność}
Analiza średniego wyniku dodatniego (wykres) wskazuje, że~dla każdego z~rozważanych poziomów składki przeciętna wysokość nadwyżki w~scenariuszach, gdzie firma pozostaje wypłacalna, przekracza poziom wyjściowy, tj.~10~000~PLN. Oznacza~to, że~działalność jest bezsprzecznie dochodowa pod~warunkiem ograniczenia ryzyka bankructwa do~akceptowalnego poziomu. Przykładowo dla~składki równej 1000~PLN, wyłączając przypadki niewypłacalności, nadwyżka końcowa jest ponad czterokrotnie większa od~początkowej. Z~kolei dla~rekomendowanej wysokości składki (1500~PLN) pozytywne scenariusze kończyły~się nadwyżką wynoszącą przeciętnie niespełna 90~000~PLN.

\section{Analiza wrażliwości}

\section{Wnioski i zalecenia}
%\bibliography{•}
\end{document}