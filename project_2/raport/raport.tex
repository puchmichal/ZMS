\documentclass[12pt, a4paper, oneside]{mwart} % Z automatu 10pt w przypisach
\usepackage[utf8]{inputenc} % Znaki diakrytyczne z klawiatury
\usepackage[OT4]{fontenc} % OT4 ponizej nie dzialalo
\usepackage[plmath,MeX]{polski} % Ponoc lepsza polonizacja LaTeXa
%\usepackage[dvips]{graphicx}
\usepackage[pdftex]{color,graphicx} % Grafika w PDFowej formie
%\usepackage{dcolumn} % Wyrownywanie przecinka w tabelach
%\newcolumntype{d}[1]{D{.}{,}{#1}} % Typ kolumny do wyrownywania
%\usepackage{threeparttable} % Coby ladnie podpisac tabelki
%\usepackage{rotating} % for sidewaystable
%\usepackage{subfig}

\usepackage[pdftex]{hyperref} % Zarzadza hiperlaczami w dokumencie, ostatni w preambule, dvips/pdftex zaleznie od wyjscia

\begin{document}
\title{\includegraphics[width = 0.3 \textwidth]{wykresy/SGHlogotypCMYKpl.eps}\\
\bigskip
Zaawansowane Modelowanie Symulacyjne [234060-0723]\\ 
\bigskip
GWINTEX S.A.\\
Symulacja linii produkującej korkociągi}
\author{Anna Chojnacka, 68729 \and
Michał Puchalski, 67827 \and
Paweł Sadłowski, 68404 }
\date{Warszawa, 7.05.2019}
\maketitle

\pagebreak

\begin{abstract}
Tutaj executive summary na koniec.
\end{abstract}

\pagebreak

\section{Opis~organizacji}
GWINTEX S.A. to międzynarodowa organizacja zajmująca się produkcją korkociągów. Firma jest w pionierem w~wykorzystaniu najnowszych technologii i~zaawansowanych maszyn metalurgicznych. Już w~3~lata po~rozpoczęciu działalności jako niewielkie rodzinne przedsiębiorstwo GWINTEX~S.A. stał~się największym pracodawcą w~powiecie oraz jednym z~trzech najważniejszych graczy w~branży. Oferta obejmuje szeroką gamę korkociągów: od~najprostszych modeli wykonanych z~metalu bądź aluminium po kunsztownie zdobione egzemplarze. Aktualnie w~związku z~pokaźnym zasobem oszczędności oraz rosnącym portfelem zamówień zza granicy przedsiębiorstwo zamierza przeprowadzić ekspansję.

\section{Opis problemu}
W związku z~rosnącym popytem i~coraz większą liczbą spływających zamówień zarząd firmy GWINTEX~S.A. planuje wybudować nową halę produkcyjną, aby zwiększyć moce przerobowe. Celem prowadzonej symulacji jest ustalenie, jaki będzie najefektywniejszy układ maszyn w nowej placówce oraz ile pakietów narzędzi naprawczych należy zakupić. Maszyny, które zarząd planuje nabyć, mają tendencję do psucia się, konieczne będzie zatem znalezienie rozwiązania minimalizującego czas, w którym maszyny nie pracują, przy uwzględnieniu kosztów. Dodatkowo została przeprowadzona analiza wrażliwości obejmująca zabezpieczenie się na wypadek czarnego scenariusza w postaci spadku popytu w przyszłości oraz analiza opłacalności szkoleń dla pracowników, które pomogłyby skrócić czas naprawy psujących się maszyn.

\subsection{Szczegółowy scenariusz symulacji}

Firma GWINTEX~S.A. dysponuje historycznymi danymi dotyczącymi bezawaryjnego czasu pracy maszyn --- ma on rozkład wykładniczy z wartością oczekiwaną równą 75 minut. Wiemy również, że każda maszyna ma przypisanego operatora, który obsługuje maszynę i~ją naprawia. Na podstawie tych samych danych wiadomo, że czas naprawy jest zmienną losową z rozkładu Erlanga, gdzie k=3 a wartość oczekiwana wynosi 15 minut. Co ważne, liczba zestawów narzędzi do naprawy maszyn znajdujących się w~zakładzie jest ograniczona. W momencie, w którym występuje usterka, zestaw zostaje wysłany z magazynu do operatora, po ukończonej naprawie wraca do niego z~powrotem. Dopiero po zakończeniu całego procesu narzędzia mogą zostać ponownie wysłane do kolejnego operatora. We wszystkich analizach przyjęliśmy 30-dniowy horyzont czasowy do oszacowania kosztów związanych z~każdym ze scenariuszy. Dla zapewnienia stabilności wyników wygenerowano 1000 symulacji dla każdego scenariusza.

\subsection{Struktura modelu}
Zgodnie z~terminologią zastosowaną przez Averilla Lawa (\cite{law}) zastosowany model jest przykładem symulacji zdarzeń dyskretnych (discrete-event simulation). Modelowanymi zjawiskami są: moment wystąpienia awarii dla każdej z maszyn (rozkład wykładniczy) oraz czas naprawy (rozkład Erlanga). Wartości tych zmiennych są losowane niezależnie. Czas oczekiwania na narzędzia oraz czas naprawy maszyny składają się następnie na~czas przestoju maszyny. W~każdej iteracji symulowany jest cały horyzont czasowy (30~dni) dla wszystkich maszyn. Ostateczny wynik jest średnią łącznego czasu przestoju danej maszyny w~kolejnych iteracjach i w~razie potrzeby może być dalej agregowany.



\begin{thebibliography}{9}
%\bibitem{slajdy}
%P.~Szufel, \emph{Zaawansowane Modelowanie Symulacyjne --- materiały do~wykładu}
\bibitem{law}
Averill~M.~Law, W.~David~Kelton,
\emph{Simulation Modeling \& Analysis},
McGraw-Hill, wyd.~drugie, 1991
\bibitem{case}
P.~Wojewnik, \emph{Pitu case study}
\bibitem{mielczarek}
Bożena~Mielczarek, \emph{Modelowanie symulacyjne w~zarządzaniu. Symulacja dyskretna},
Oficyna Wydawnicza Politechniki Wrocławskiej, Wrocław~2009
\end{thebibliography}

\end{document}